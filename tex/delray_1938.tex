\documentclass{article}
\usepackage[utf8]{inputenc}

\title{Helen O'Loy}
\author{Lester del Rey}
\date{Originally published in \par \textit{Astounding Science Fiction}, December 1938.}

\begin{document}

\maketitle

I am an old man now, but I can still see Helen as Dave unpacked her, and still hear him gasp as he looked her over.

“Lord, isn’t she a beauty?”

She was beautiful, a dream in spun plastics and metals, something Keats might have seen dimly when he wrote sonnets. If Helen of Troy had looked like that the Greeks must have been pikers when they launched only a thousand ships; at least, that’s what I told Dave.

“Helen of Troy, eh?” He looked at her tag. “At least it beats this thing—K2W88. Helen… Mmmm… Helen of Alloy.”

“Not much swing to that, Dave. Too many unstressed syllables in the middle. How about Helen O’Loy?”

“Helen O’Loy she is, Phil.”

And that’s how it began—one part beauty, one part dream, one part science; add a stereo broadcast, stir mechanically, and the result is chaos. Dave and I hadn’t gone to college together, but when I came to Messina to practice medicine, I found him downstairs in a little robot repair shop. After that, we began to pal around, and when I started going with one twin, he found the other equally attractive, so we made it a foursome.

When our business grew better, we rented a house out near the rocket field—noisy but cheap, and the rockets discouraged apartment building. We liked room enough to stretch ourselves. I suppose, if we hadn’t quarreled with them, we’d have married the twins in time. But Dave wanted to look over the latest Venus-rocket attempt when his twin wanted to see a display stereo starring Larry Ainslee, and they were both stubborn. From then on, we forgot the twins and spent our evenings at home.

But it wasn’t until “Lenny” put vanilla on our steak instead of salt that we got off on the subject of emotions and robots. While Dave was dissecting Lenny to find the trouble, we naturally mulled over the future of the mechs. He was sure that the robots would beat humans some day, and I couldn’t see it.

“Look here, Dave,” I argued. “You know Lenny doesn’t think—not really. When those wires crossed, she could have corrected herself. But she didn’t bother; she followed the mechanical impulse. A woman might have reached for the vanilla, but when she saw it in her hand, she’d have stopped. Lenny has sense enough, but she has no emotions, no consciousness of self.”

“All right, that’s the big trouble with the mechs now. But we’ll get around it, put in some mechanical emotions, or something.” He screwed Lenny’s head back on, turned on her juice. “Go back to work, Lenny, it’s nineteen o’clock.”

Now I specialized in endocrinology and related subjects. I wasn’t exactly a psychologist, but I did understand the glands, secretions, hormones, and miscellanies that are the physical causes of emotions. It took medical science three hundred years to find out how and why they worked, and I couldn’t see people duplicating them mechanically in much less time. I brought home books and papers to prove it, and Dave quoted the invention of memory coils and veritoid eyes. During that year we swapped knowledge until Dave knew the whole theory of endocrinology, and I could have made Lenny from memory. The more we talked, the less sure I grew about the impossibility of Homo mechanensis as the perfect type.

Poor Lenny. Her cuproberyl body spent half its time in scattered pieces. Our first attempts were successful only in getting her to serve fried brushes for breakfast and wash the dishes in oleo oil. Then one day she cooked a perfect dinner with six wires crossed, and Dave was in ecstasy.

He worked all night on her wiring, put in a new coil, and taught her a fresh set of words. And the next day she flew into a tantrum and swore vigorously at us when we told her she wasn’t doing her work right.

“It’s a lie,” she yelled, shaking a suction brush. “You’re all liars. If you so-and-so’s would leave me whole long enough, I might get something done around the place.”

When we calmed her temper and got her back to work, Dave ushered me into the study. “Not taking any chances with Lenny,” he explained. “We’ll have to cut out that adrenal pack and restore her to normality. But we’ve got to get a better robot. A housemaid mech isn’t complex enough.”

“How about Dillard’s new utility models? They seem to combine everything in one.”

“Exactly. Even so, we’ll need a special one built to order, with a full range of memory coils. And out of respect to old Lenny, let’s get a female case for its works.”

The result, of course, was Helen. The Dillard people had performed a miracle and put all the works in a girl-modeled case. Even the plastic and rubberite face was designed for flexibility to express emotions, and she was complete with tear glands and taste buds, ready to simulate every human action, from breathing to pulling hair, the bill they sent with her was another miracle, but Dave and I scraped it together; we had to turn Lenny over to an exchange to complete it, though, and thereafter we ate out.

I’d performed plenty of delicate operations on living tissues, and some of them had been tricky, but I still felt like a premed student as we opened the front plate of her torso and began to sever the leads of her “nerves.” Dave’s mechanical glands were all prepared, complex little bundles of pansistors and wires that heterodyned on the electrical thought impulses and distorted them as adrenalin distorts the reaction of human minds.

Instead of sleeping that night, we pored over the schematic diagrams of her structures, tracing the complex thought mazes of her wiring, severing the leaders, implanting the heterones, as Dave called them. And while we worked, a mechanical tape fed carefully prepared thoughts of consciousness and awareness of life and feeling into an auxiliary memory coil. Dave believed in leaving nothing to chance.

It was growing light as we finished, exhausted and exultant. All that remained was the starting of her electrical power; like all the Dillard mechs, she was equipped with a tiny atomotor instead of batteries, and once started would need no further attention.

Dave refused to turn her on. “Wait until we’ve slept and rested,” he advised. “I’m as eager to try her as you are, but we can’t do much studying with our minds half-dead. Turn in, and we’ll leave Helen until later.”

Even though we were both reluctant to follow it, we knew the idea was sound. We turned in, and sleep hit us before the air conditioner could cut down to sleeping temperature. And then Dave was pounding on my shoulder.

“Phil! Hey, snap out of it!”

I groaned, turned over, and faced him. “Well…? Uh! What is it? Did Helen—”

“No, it’s old Mrs. van Styler. She ’visored to say her son has an infatuation for a servant, and she wants you to come out and give counterhormones. They’re at the summer camp in Maine.”

Rich Mrs. van Styler! I couldn’t afford to let that account down, now that Helen had used up the last of my funds. But it wasn’t a job I cared for.

“Counterhormones! That’ll take two weeks’ full time. Anyway, I’m no society doctor, messing with glands to keep fools happy. My job’s taking care of serious trouble.”

“And you want to watch Helen.” Dave was grinning, but he was serious, too. “I told her it’d cost her fifty thousand!”

“Huh?”

“And she said okay, if you hurried.”

Of course, there was only one thing to do, though I could have wrung fat Mrs. van Styler’s neck cheerfully. It wouldn’t have happened if she’d used robots like everyone else—but she had to be different.

Consequently, while Dave was back home puttering with Helen, I was racking my brain to trick Archy van Styler into getting the counterhormones, and giving the servant the same. Oh, I wasn’t supposed to, but the poor kid was crazy about Archy. Dave might have written, I thought, but never a word did I get.

It was three weeks later instead of two when I reported that Archy was “cured,” and collected on the line. With that money in my pocket, I hired a personal rocket and was back in Messina in half an hour. I didn’t waste time in reaching the house.

As I stepped into the alcove, I heard a light patter of feet, and an eager voice called out, “Dave, dear?” For a minute I couldn’t answer, and the voice came again, pleading, “Dave?”

I don’t know what I expected, but I didn’t expect Helen to meet me that way, stopping and staring at me, obvious disappointment on her face, little hands fluttering up against her chest.

“Oh,” she cried. “I thought it was Dave. He hardly comes home to eat now, but I’ve had supper waiting hours.” She dropped her hands and managed a smile. “You’re Phil, aren’t you? Dave told me about you when…at first. I’m so glad to see you home, Phil.”

“Glad to see you doing so well, Helen.” Now what does one say for light conversation with a robot? “You said something about supper?”

“Oh, yes. I guess Dave ate downtown again, so we might as well go in. It’ll be nice having someone to talk to around the house, Phil. You don’t mind if I call you Phil, do you? You know, you’re sort of a godfather to me.”

We ate. I hadn’t counted on such behavior, but apparently she considered eating as normal as walking.

She didn’t do much eating, at that; most of the time she spent staring at the front door.

Dave came in as we were finishing, a frown a yard wide on his face. Helen started to rise, but he ducked toward the stairs, throwing words over his shoulder. “Hi, Phil. See you up here later.” There was something radically wrong with him. For a moment, I’d thought his eyes were haunted, and as I turned to Helen, hers were filling with tears. She gulped, choked them back, and fell to viciously on her food.

“What’s the matter with him…and you?” I asked.

“He’s sick of me.” She pushed her plate away and got up hastily. “You’d better see her while I clean up. And there’s nothing wrong with me. And it’s not my fault anyway.” She grabbed the dishes and ducked into the kitchen; I could have sworn she was crying.

Maybe all thought is a series of conditioned reflexes—but she certainly had picked up a lot of conditioning while I was gone. Lenny in her heyday had been nothing like this. I went up to see if Dave could make any sense out of the hodge-podge.

He was squirting soda into a large glass of apple brandy, and I saw that the bottle was nearly empty. “Join me?” he asked.

It seemed like a good idea. The roaring blast of an ion rocket overhead was the only familiar thing left in the house. From the look around Dave’s eyes, it wasn’t the first bottle he’d emptied while I was gone, and there were more left. He dug out a new bottle for his own drink.

“Of course, it’s none of my business, Dave, but that stuff won’t steady your nerves any. What’s gotten into you and Helen? Been seeing ghosts?”

Helen was wrong; he hadn’t been eating downtown—nor anywhere else. His muscles collapsed into a chair in a way that spoke of fatigue and nerves, but mostly of hunger. “You noticed it, eh?”

“Noticed it? The two of you jammed it down my throat.”

“Uhmmm.” He swatted at a nonexistent fly, and slumped further down in the pneumatic. “Guess maybe I should have waited with Helen until you got back. But if that stereo cast hadn’t changed…anyway, it did. And those mushy books of yours finished the job.”

“Thanks. That makes it all clear.”

“You know, Phil, I’ve got a place up in the country…fruit ranch. My dad left it to me. Think I’ll look it over.”

And that’s the way it went. But finally, by much liquor and more perspiration, I got some of the story out of him before I gave him an Amytal and put him to bed. Then I hunted up Helen and dug the rest of the story from her, until it made sense.

Apparently as soon as I was gone, Dave had turned her on and made preliminary tests, which were entirely satisfactory. She had reacted beautifully—so well that he decided to leave her and go down to work as usual.

Naturally, with all her untried emotions, she was filled with curiosity, and wanted him to stay. Then he had an inspiration. After showing her what her duties about the house would be, he set her down in front of the stereovisor, tuned in a travelogue, and left her to occupy her time with that.

The travelogue held her attention until it was finished, and the station switched over to a current serial with Larry Ainslee, the same cute emoter who’d given us all the trouble with the twins. Incidentally, he looked something like Dave.

Helen took to the serial like a seal to water. This play-acting was a perfect outlet for her newly excited emotions. When that particular episode finished, she found a love story on another station, and added still more to her education. The afternoon programs were mostly news and music, but by then she’d found my books; and I do have rather adolescent taste in literature.

Dave came home in the best of spirits. The front alcove was neatly swept, and there was the odor of food in the air that he’d missed around the house for weeks. He had visions of Helen as the super-efficient housekeeper.

So it was a shock to him to feel two strong arms around his neck from behind and hear a voice all aquiver coo into his ears, “Oh, Dave, darling. I’ve missed you so, and I’m so thrilled that you’re back.” Helen’s technique may have lacked polish, but it had enthusiasm, as he found when he tried to stop her from kissing him. She had learned fast and furiously—also, Helen was powered by an atomotor.

Dave wasn’t a prude, but he remembered that she was only a robot, after all. The fact that she felt, acted, and looked like a young goddess in his arms didn’t mean much. With some effort, he untangled her and dragged her off to supper, where he made her eat with him to divert her attention.

After her evening work, he called her into the study and gave her a thorough lecture on the folly of her ways. It must have been good, for it lasted three solid hours, and covered her station in life, the idiocy of stereos, and various other miscellanies. When he finished, Helen looked up with dewy eyes and said wistfully, “I know, Dave, but I still love you.” That’s when Dave started drinking. It grew worse each day. If he stayed downtown, she was crying when he came home. If he returned on time, she fussed over him and threw herself at him. In his room, with the door locked, he could hear her downstairs pacing up and down and muttering; and when he went down, she stared at him reproachfully until he had to go back up.

I sent Helen out on a fake errand in the morning and got Dave up. With Helen gone, I made Dave eat a decent breakfast and gave him a tonic for his nerves. He was still listless and moody.

“Look here, Dave,” I broke in on his brooding. “Helen isn’t human, after all. Why not cut off her power and change a few memory coils? Then we can convince her that she never was in love and couldn’t get that way.”

“You try it. I had that idea, but she put up a wail that would wake Homer. She says it would be murder—and the hell of it is that I can’t help feeling the same about it. Maybe she isn’t human, but you wouldn’t guess it when she puts on that martyred look and tells you to go ahead and kill her.”

“We never put in substitutes for some of the secretions present in human during the love period.”

“I don’t know what we put in. Maybe the heterones backfired or something. Anyway, she’s made this idea so much a part of her thoughts that we’d have to put in a whole new set of coils.”

“Well, why not?”

“Go ahead. You’re the surgeon of this family. I’m not used to fussing with emotions. Matter of fact, since she’s been acting this way, I’m beginning to hate work on any robot. My business is going to blazes.”

He saw Helen coming up the walk and ducked out the back door for the monorail express. I’d intended to put him back in bed, but let him go. Maybe he’d be better off at his shop than at home.

“Dave’s gone?” Helen did have that martyred look now. “Yeah. I got him to eat, and he’s gone to work.”

“I’m glad he ate.” She slumped down in a chair as if she were worn out, though how a mech could be tired beat me. “Phil?”

“Well, what is it?”

“Do you think I’m bad for him? I mean, do you think he’d be happier if I weren’t here?”

“He’ll go crazy if you keep acting this way around him.”

She winced. Those little hands were twisting about pleadingly, and I felt like an inhuman brute. But I’d started, and I went ahead. “Even if I cut out your power and changed your coils, he’d probably still be haunted by you.”

“I know. But I can’t help it. And I’d make him a good wife, really I would, Phil.”

I gulped; this was getting a little too far. “And give him strapping sons to boot, I suppose. A man wants flesh and blood, not rubber and metal.”

“Don’t, please! I can’t think of myself that way; to me, I’m a woman. And you know how perfectly I’m made to imitate a real woman…in all ways. I couldn’t give him children, but in every other way…I’d try so hard, I know I’d make him a good wife.”

I gave up.

Dave didn’t come home that night, nor the next day. Helen was fussing and fuming, wanting me to call the hospitals and the police, but I knew nothing had happened to him. He always carried identification. Still, when he didn’t come on the third day, I began to worry. And when Helen started out for his shop, I agreed to go with her.

Dave was there, with another man I didn’t know. I parked Helen where he couldn’t see her, but where she could hear, and went in as soon as the other fellow left.

Dave looked a little better and seemed glad to see me. “Hi, Phil—just closing up. Let’s go eat.”

Helen couldn’t hold back any longer, but came trooping in. “Come on home, Dave. I’ve got roast duck with spice stuffing, and you know you love that.”

“Scat!” said Dave. She shrank back, turned to go. “Oh, all right, stay. You might as well hear it, too. I’ve sold the shop. The fellow you saw just bought it, and I’m going up to the old fruit ranch I told you about, Phil. I can’t stand the mechs any more.”

“You’ll starve to death at that,” I told him.

“No, there’s a growing demand for old-fashioned fruit, raised out of doors. People are tired of this water-culture stuff. Dad always made a living out of it. I’m leaving as soon as I can get home and pack.”

Helen clung to her idea. “I’ll pack, Dave, while you eat. I’ve got apple cobbler for dessert.” The world was toppling under her feet, but she still remembered how crazy he was for apple cobbler.

Helen was a good cook; in fact she was a genius, with all the good points of a woman and a mech combined. Dave ate well enough, after he got started. By the time supper was over, he’d thawed out enough to admit he liked the duck and cobbler, and to thank her for packing. In fact, he even let her kiss him good-bye, though he firmly refused to let her go to the rocket field with him.

Helen was trying to be brave when I got back, and we carried on a stumbling conversation about Mrs. van Styler’s servants for a while. But the talk began to lull, and she sat staring out of the window at nothing most of the time. Even the stereo comedy lacked interest for her, and I was glad enough to have her go off to her room. She could cut her power down to simulate sleep when she chose.

As the days slipped by, I began to realize why she couldn’t believe herself a robot. I got to thinking of her as a girl and companion myself. Except for odd intervals when she went off by herself to brood, or when she kept going to the telescript for a letter that never came, she was as good a companion as a man could ask. There was something homey about the place that Lenny had never put there.

I took Helen on a shopping trip to Hudson and she laughed and purred over the wisps of silk and glassheen that were the fashion, tried on endless hats, and conducted herself as any normal girl might. We went trout fishing for a day, where she proved to be as good a sport and sensibly silent. I thoroughly enjoyed myself and thought she was forgetting Dave. That was before I came home unexpectedly and found her doubled up on the couch, threshing her legs up and down and crying to the high heavens.

It was then I called Dave. They seemed to have trouble in reaching him, and Helen came over beside me while I waited. She was tense and fidgety as an old maid trying to propose. But finally they located Dave.

“What’s up, Phil?” he asked as his face came on the viewplate. “I was just getting my things together to—”

I broke him off. “Things can’t go on the way they are, Dave. I’ve made up my mind. I’m yanking Helen’s coils tonight. It won’t be worse than what she’s going through now.”

Helen reached up and touched my shoulder. “Maybe that’s best, Phil. I don’t blame you.”

Dave’s voice cut in. “Phil, you don’t know what you’re doing!”

“Of course, I do. It’ll all be over by the time you can get here. As you heard, she’s agreeing.”

There was a black cloud sweeping over Dave’s face. “I won’t have it, Phil. She’s half mine and I forbid it!”

“Of all the—”

“Go ahead, call me anything you want. I’ve changed my mind. I was packing to come home when you called.”

Helen jerked around me, her eyes glued to the panel. “Dave, do you…are you—”

“I’m just waking up to what a fool I’ve been, Helen. Phil, I’ll be home in a couple of hours, so if there’s anything—”

He didn’t have to chase me out. But I heard Helen cooing something about loving to be a rancher’s wife before I could shut the door.

Well, I wasn’t as surprised as they thought. I think I knew when I called Dave what would happen. No man acts the way Dave had been acting because he hates a girl; only because he thinks he does—and thinks wrong.

No woman ever made a lovelier bride or a sweeter wife. Helen never lost her flair for cooking and making a home. With her gone, the old house seemed empty, and I began to drop out to the ranch once or twice a week. I suppose they had trouble at times, but I never saw it, and I know the neighbors never suspected they were anything but a normal couple.

Dave grew older, and Helen didn’t, of course. But between us, we put lines in her face and grayed her hair without letting Dave know that she wasn’t growing old with him; he’d forgotten that she wasn’t human, I guess.

I practically forgot, myself. It wasn’t until a letter came from Helen this morning that I woke up to reality. There, in her beautiful script, just a trifle shaky in places, was the inevitable that neither Dave nor I had seen.

\begin{quote}

Dear Phil,

As you know, Dave has had heart trouble for several years now. We expected him to live on just the same, but it seems that wasn’t to be. He died in my arms just before sunrise. He sent you his greetings and farewell.

I’ve one last favor to ask of you, Phil. There is only one thing for me to do when this is finished. Acid will burn out metal as well as flesh, and I’ll be dead with Dave. Please see that we are buried together, and that the morticians do not find my secret. Dave wanted it that way, too.

Poor, dear Phil. I know you loved Dave as a brother, and how you felt about me. Please don’t grieve too much for us, for we have had a happy life together, and both feel that we should cross this last bridge side by side.

With love and thanks from

Helen

\end{quote}

It had to come sooner or later, I suppose, and the first shock has worn off now. I’ll be leaving in a few minutes to carry out Helen’s last instructions.

Dave was a lucky man, and the best friend I ever had. And Helen—well, as I said, I’m an old man now, and can view things more sanely, I should have married and raised a family, I suppose. But…there was only one Helen O’Loy.

\end{document}